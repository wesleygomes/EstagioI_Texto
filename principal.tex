% verso e anverso:
\documentclass[12pt,openright,twoside,a4paper,english,french,spanish,brazil]{abntex2}	

% apenas verso: 
%\documentclass[12pt,oneside,a4paper,english,french,spanish,brazil]{abntex2}

% ---
% PACOTES
% ---

% ---
% Pacotes fundamentais 
% ---
\usepackage{cmap}				% Mapear caracteres especiais no PDF
\usepackage{lmodern}			% Usa a fonte Latin Modern
\usepackage[T1]{fontenc}		% Selecao de codigos de fonte.
\usepackage[utf8]{inputenc}		% Codificacao do documento (conversão automática dos acentos)
\usepackage{indentfirst}		% Indenta o primeiro parágrafo de cada seção.
\usepackage{color}				% Controle das cores
\usepackage{graphicx}			% Inclusão de gráficos
% ---

% ---
% Pacotes adicionais, usados no anexo do modelo de folha de identificação
% ---
\usepackage{multicol}
\usepackage{multirow}
% ---
	
% ---
% Pacotes adicionais, usados apenas no âmbito do Modelo Canônico do abnteX2
% ---
\usepackage{lipsum}				% para geração de dummy text
% ---

% ---
% Pacotes de citações
% ---
\usepackage[brazilian,hyperpageref]{backref}	 % Paginas com as citações na bibl
\usepackage[alf]{abntex2cite}	% Citações padrão ABNT

% --- 
% CONFIGURAÇÕES DE PACOTES
% --- 

% ---
% Configurações do pacote backref
% Usado sem a opção hyperpageref de backref
\renewcommand{\backrefpagesname}{Citado na(s) página(s):~}
% Texto padrão antes do número das páginas
\renewcommand{\backref}{}
% Define os textos da citação
\renewcommand*{\backrefalt}[4]{
	\ifcase #1 %
		Nenhuma citação no texto.%
	\or
		Citado na página #2.%
	\else
		Citado #1 vezes nas páginas #2.%
	\fi}%
% ---

% ---
% Informações de dados para CAPA e FOLHA DE ROSTO
% ---
\titulo{ESTUDO DE DESENVOLVIMENTO HÍBRIDO PARA UM SISTEMA DE CONTROLE DE VENDAS PET SHOP}
\autor{Wesley Gomes da Silva}
\local{Palmas-TO}
\data{2013}
\instituicao{%
  Faculdade Católica do Tocantins
  \par
  Curso Sistemas de Informação
  }
\tipotrabalho{Relatório Final de Estágio}
% O preambulo deve conter o tipo do trabalho, o objetivo, 
% o nome da instituição e a área de concentração 
\preambulo{Projeto apresentado como requisito parcial para aprovação na disciplina de Estágio Supervisionado I do Curso de Sistemas de Informação, da Faculdade Católica do Tocantins (FACTO), sob a orientação do professor Ms. Marco Antônio Firmino de Sousa.}
% ---

% ---
% Configurações de aparência do PDF final

% alterando o aspecto da cor azul
\definecolor{blue}{RGB}{41,5,195}

% informações do PDF
\makeatletter
\hypersetup{
     	%pagebackref=true,
		pdftitle={\@title}, 
		pdfauthor={\@author},
    	pdfsubject={\imprimirpreambulo},
	    pdfcreator={LaTeX with abnTeX2},
		pdfkeywords={abnt}{latex}{abntex}{abntex2}{relatório técnico}, 
		colorlinks=true,       		% false: boxed links; true: colored links
    	linkcolor=blue,          	% color of internal links
    	citecolor=blue,        		% color of links to bibliography
    	filecolor=magenta,      		% color of file links
		urlcolor=blue,
		bookmarksdepth=4
}
\makeatother
% --- 

% --- 
% Espaçamentos entre linhas e parágrafos 
% --- 

% O tamanho do parágrafo é dado por:
\setlength{\parindent}{1.3cm}

% Controle do espaçamento entre um parágrafo e outro:
\setlength{\parskip}{0.2cm}  % tente também \onelineskip

% ---
% compila o indice
% ---
\makeindex
% ---

% ----
% Início do documento
% ----
\begin{document}

% Retira espaço extra obsoleto entre as frases.
\frenchspacing 

% ----------------------------------------------------------
% ELEMENTOS PRÉ-TEXTUAIS
% ----------------------------------------------------------
% \pretextual

% ---
% Capa
% ---
\imprimircapa
% ---

% ---
% Folha de rosto
% (o * indica que haverá a ficha bibliográfica)
% ---
\imprimirfolhaderosto*
% ---

% ---
% Anverso da folha de rosto:
% ---

{
\ABNTEXchapterfont

\vspace*{\fill}

Conforme a ABNT NBR 10719:2011, seção 4.2.1.1.1, o anverso da folha de rosto
deve conter:

\begin{alineas}
  \item nome do órgão ou entidade responsável que solicitou ou gerou o
   relatório; 
  \item título do projeto, programa ou plano que o relatório está relacionado;
  \item título do relatório;
  \item subtítulo, se houver, deve ser precedido de dois pontos, evidenciando a
   sua subordinação ao título. O relatório em vários volumes deve ter um título
   geral. Além deste, cada volume pode ter um título específico; 
  \item número do volume, se houver mais de um, deve constar em cada folha de
   rosto a especificação do respectivo volume, em algarismo arábico; 
  \item código de identificação, se houver, recomenda-se que seja formado
   pela sigla da instituição, indicação da categoria do relatório, data,
   indicação do assunto e número sequencial do relatório na série; 
  \item classificação de segurança. Todos os órgãos, privados ou públicos, que
   desenvolvam pesquisa de interesse nacional de conteúdo sigiloso, devem
    informar a classificação adequada, conforme a legislação em vigor; 
  \item nome do autor ou autor-entidade. O título e a qualificação ou a função
   do autor podem ser incluídos, pois servem para indicar sua autoridade no
   assunto. Caso a instituição que solicitou o relatório seja a mesma que o
   gerou, suprime-se o nome da instituição no campo de autoria; 
  \item local (cidade) da instituição responsável e/ou solicitante; NOTA: No
   caso de cidades homônimas, recomenda-se o acréscimo da sigla da unidade da
   federação.
  \item ano de publicação, de acordo com o calendário universal (gregoriano),
  deve ser apresentado em algarismos arábicos.
\end{alineas}

\vspace*{\fill}
}

% ---
% Agradecimentos
% ---
\begin{agradecimentos}
Agradeço primeiramente a Deus pela vida, ao professor Marco Antonio Firmino de Sousa pela orientação e motivação para realizar este trabalho, agradeço em especial a minha família por estar ao meu lado na busca por este sonho e também meus amigos pelo incentivo na realização deste.

\end{agradecimentos}
% ---

% ---
% RESUMO
% ---

% resumo na língua vernácula (obrigatório)
\begin{resumo}
 Segundo a \citeonline[3.1-3.2]{NBR6028:2003}, o resumo deve ressaltar o
 objetivo, o método, os resultados e as conclusões do documento. A ordem e a extensão
 destes itens dependem do tipo de resumo (informativo ou indicativo) e do
 tratamento que cada item recebe no documento original. O resumo deve ser
 precedido da referência do documento, com exceção do resumo inserido no
 próprio documento. (\ldots) As palavras-chave devem figurar logo abaixo do
 resumo, antecedidas da expressão Palavras-chave:, separadas entre si por
 ponto e finalizadas também por ponto.

 \vspace{\onelineskip}
    
 \noindent
 \textbf{Palavras-chaves}: latex. abntex. editoração de texto.
\end{resumo}
% ---

% ---
% inserir lista de ilustrações
% ---
\pdfbookmark[0]{\listfigurename}{lof}
\listoffigures*
\cleardoublepage
% ---

% ---
% inserir lista de tabelas
% ---
\pdfbookmark[0]{\listtablename}{lot}
\listoftables*
\cleardoublepage
% ---

% ---
% inserir lista de abreviaturas e siglas
% ---
\begin{siglas}
  \item[Fig.] Area of the $i^{th}$ component
  \item[456] Isto é um número
  \item[123] Isto é outro número
  \item[lauro cesar] este é o meu nome
\end{siglas}
% ---

% ---
% inserir lista de símbolos
% ---
\begin{simbolos}
  \item[$ \Gamma $] Letra grega Gama
  \item[$ \Lambda $] Lambda
  \item[$ \zeta $] Letra grega minúscula zeta
  \item[$ \in $] Pertence
\end{simbolos}
% ---

% ---
% inserir o sumario
% ---
\pdfbookmark[0]{\contentsname}{toc}
\tableofcontents*
\cleardoublepage
% ---


% ----------------------------------------------------------
% ELEMENTOS TEXTUAIS
% ----------------------------------------------------------
\textual

% ----------------------------------------------------------
% Introdução
% ----------------------------------------------------------
\chapter*[Introdução]{Introdução}
\addcontentsline{toc}{chapter}{Introdução}

Nos últimos anos, com o advento dos dispositivos portáteis a nível mundial, houve um crescimento gigantesco de aplicações web mobile, principalmente no meio coorporativo, na utilização de soluções móveis. Devido a tamanha evolução, novas plataformas de desenvolvimento e sistemas operacionais foram criados, aumentando a complexidade e muitas vezes a curva de aprendizado. Um dos maiores problemas com a disparidade entre plataformas é, a manutenção da aplicação, controle de atualizações e custeamento de equipes de desenvolvimento com conhecimentos específicos, como: Android, iOS, BlackBerry, Windows Phone, etc.
As WebApps, são aplicações projetadas para serem executadas em browsers de dispositivos móveis, que acoplada em novas técnicas de desenvolvimento com determinados frameworks estão se tornando os nomeados Web Apps Híbridos. Ou seja, a sua interface gráfica é adaptada para dispositivos com telas menores, utilizando conceitos como o responsive (são web sites que através de técnicas de construção do HTML5 e CSS, é possível termos a versão de um site que se modifique conforme o dispositivo utilizadocom excelente visualização em plataformas e resoluções diferentes). Estas aplicações podem ser acessado de qualquer computador ou dispositivo móvel por meio de um browser conectado à internet, na maior parte, são hospedadas em servidores web, utilizando tecnologias específicas para serem carregadas, em máquinas de “baixo” processamento e normalmente com baixa velocidade de banda da rede.


% ----------------------------------------------------------
% PARTE - preparação da pesquisa
% ----------------------------------------------------------
\part{Preparação do relatório}

% ----------------------------------------------------------
% Capitulo com exemplos de comandos inseridos de arquivo externo 
% ----------------------------------------------------------

\include{abntex2-modelo-include-comandos}


% ----------------------------------------------------------
% Parte de revisãod e literatura
% ----------------------------------------------------------
\part{Resultados}

% ---
% Capitulo de revisão de literatura
% ---
\chapter{Lorem ipsum dolor sit amet}

% ---
\section{Aliquam vestibulum fringilla lorem}
% ---

\lipsum[1]

\lipsum[2-3]

% ---
% Finaliza a parte no bookmark do PDF, para que se inicie o bookmark na raiz
% ---
\bookmarksetup{startatroot}% 
% ---

% ---
% Conclusão
% ---
\chapter*[Conclusão]{Conclusão}
\addcontentsline{toc}{chapter}{Conclusão}

\lipsum[31-33]

% ----------------------------------------------------------
% ELEMENTOS PÓS-TEXTUAIS
% ----------------------------------------------------------
\postextual

% ----------------------------------------------------------
% Referências bibliográficas
% ----------------------------------------------------------
\bibliography{abntex2-modelo-references}

% ----------------------------------------------------------
% Glossário
% ----------------------------------------------------------
%
% Consulte o manual da classe abntex2 para orientações sobre o glossário.
%
%\glossary

% ----------------------------------------------------------
% Apêndices
% ----------------------------------------------------------

% ---
% Inicia os apêndices
% ---
\begin{apendicesenv}

% Imprime uma página indicando o início dos apêndices
\partapendices

% ----------------------------------------------------------
\chapter{Quisque libero justo}
% ----------------------------------------------------------

\lipsum[50]

% ----------------------------------------------------------
\chapter{Nullam elementum urna vel imperdiet sodales elit ipsum pharetra ligula
ac pretium ante justo a nulla curabitur tristique arcu eu metus}
% ----------------------------------------------------------
\lipsum[55-57]

\end{apendicesenv}
% ---


% ----------------------------------------------------------
% Anexos
% ----------------------------------------------------------

% ---
% Inicia os anexos
% ---
\begin{anexosenv}

% Imprime uma página indicando o início dos anexos
\partanexos

% ---
\chapter{Morbi ultrices rutrum lorem.}
% ---
\lipsum[30]

% ---
\chapter{Cras non urna sed feugiat cum sociis natoque penatibus et magnis dis
parturient montes nascetur ridiculus mus}
% ---

\lipsum[31]

% ---
\chapter{Fusce facilisis lacinia dui}
% ---

\lipsum[32]

\end{anexosenv}

%---------------------------------------------------------------------
% INDICE REMISSIVO
%---------------------------------------------------------------------

\printindex

%---------------------------------------------------------------------
% Formulário de Identificação (opcional)
%---------------------------------------------------------------------
\chapter*[Formulário de Identificação]{Formulário de Identificação}
\addcontentsline{toc}{chapter}{Exemplo de Formulário de Identificação}
\label{formulado-identificacao}

Exemplo de Formulário de Identificação, compatível com o Anexo A (informativo)
da ABNT NBR 10719:2011. Este formulário não é um anexo. Conforme definido na
norma, ele é o último elemento pós-textual e opcional do relatório.

\bigskip

\begin{tabular}{|p{9cm}|p{5cm}|}
\hline
\multicolumn{2}{|c|}{\textbf{\large Dados do Relatório Técnico e/ou científico}}\\
\hline
\multirow{4}{10cm}[24pt]{Título e subtítulo}& Classificação de segurança\\
                   & \\
                   \cline{2-2}
                   & No.\\
                   & \\
				
\hline
Tipo de relatório & Data\\
\hline
Título do projeto/programa/plano & No.\\
\hline
\multicolumn{2}{|l|}{Autor(es)} \\
\hline
\multicolumn{2}{|l|}{Instituição executora e endereço completo} \\
\hline
\multicolumn{2}{|l|}{Instituição patrocinadora e endereço completo} \\
\hline
\multicolumn{2}{|l|}{Resumo}\\[3cm]
\hline
\multicolumn{2}{|l|}{Palavras-chave/descritores}\\
\hline
\multicolumn{2}{|l|}{
Edição \hfill No. de páginas \hfill No. do volume \hfill Nº de classificação \phantom{XXXX}} \\
\hline
\multicolumn{2}{|l|}{
ISSN \hfill \hfill Tiragem \hfill Preço \phantom{XXXXXXXX}} \\
\hline
\multicolumn{2}{|l|}{Distribuidor} \\
\hline
\multicolumn{2}{|l|}{Observações/notas}\\[3cm]
\hline
\end{tabular}

\end{document}