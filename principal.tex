% verso e anverso:
\documentclass[12pt,openright,twoside,a4paper,english,french,spanish,brazil]{abntex2}	

% apenas verso: 
%\documentclass[12pt,oneside,a4paper,english,french,spanish,brazil]{abntex2}

% -------------------------------------------------------------------------------------------------
% PACOTES
% Pacotes fundamentais 

\usepackage{cmap}				% Mapear caracteres especiais no PDF
\usepackage{lmodern}			% Usa a fonte Latin Modern
\usepackage[T1]{fontenc}		% Selecao de codigos de fonte.
\usepackage[utf8]{inputenc}		% Codificacao do documento (conversão automática dos acentos)
\usepackage{indentfirst}		% Indenta o primeiro parágrafo de cada seção.
\usepackage{color}				% Controle das cores
\usepackage{graphicx}			% Inclusão de gráficos
% -------------------------------------------------------------------------------------------------

% Pacotes adicionais, usados no anexo do modelo de folha de identificação
\usepackage{multicol}
\usepackage{multirow}
% -------------------------------------------------------------------------------------------------
	
% Pacotes adicionais, usados apenas no âmbito do Modelo Canônico do abnteX2
\usepackage{lipsum}				% para geração de dummy text
% -------------------------------------------------------------------------------------------------

% Pacotes de citações
\usepackage[brazilian,hyperpageref]{backref}	 % Paginas com as citações na bibl
\usepackage[alf]{abntex2cite}	% Citações padrão ABNT
% -------------------------------------------------------------------------------------------------

% CONFIGURAÇÕES DE PACOTES
% Configurações do pacote backref
% Usado sem a opção hyperpageref de backref
\renewcommand{\backrefpagesname}{Citado na(s) página(s):~}
% Texto padrão antes do número das páginas
\renewcommand{\backref}{}
% Define os textos da citação
\renewcommand*{\backrefalt}[4]{
	\ifcase #1 %
		Nenhuma citação no texto.%
	\or
		Citado na página #2.%
	\else
		Citado #1 vezes nas páginas #2.%
	\fi}%
% -------------------------------------------------------------------------------------------------


% Informações de dados para CAPA e FOLHA DE ROSTO
\titulo{ESTUDO DE DESENVOLVIMENTO HÍBRIDO PARA UM SISTEMA DE CONTROLE DE VENDAS PET SHOP}
\autor{Wesley Gomes da Silva}
\local{Palmas - TO}
\data{2013}
\instituicao{%
  Faculdade Católica do Tocantins
  \par
  Curso Sistemas de Informação
  }
\tipotrabalho{Relatório Final de Estágio}
% O preambulo deve conter o tipo do trabalho, o objetivo, 
% o nome da instituição e a área de concentração 
\preambulo{Projeto apresentado como requisito parcial para aprovação na disciplina de Estágio Supervisionado I do Curso de Sistemas de Informação, da Faculdade Católica do Tocantins (FACTO), sob a orientação do professor MsC. Marco Antônio Firmino de Sousa.}
% -------------------------------------------------------------------------------------------------


% Configurações de aparência do PDF final

% alterando o aspecto da cor azul
\definecolor{blue}{RGB}{41,5,195}

% informações do PDF
\makeatletter
\hypersetup{
     	%pagebackref=true,
		pdftitle={\@title}, 
		pdfauthor={\@author},
    	pdfsubject={\imprimirpreambulo},
	    pdfcreator={LaTeX with abnTeX2},
		pdfkeywords={abnt}{latex}{abntex}{abntex2}{relatório técnico}, 
		colorlinks=true,       		% false: boxed links; true: colored links
    	linkcolor=blue,          	% color of internal links
    	citecolor=blue,        		% color of links to bibliography
    	filecolor=magenta,      		% color of file links
		urlcolor=blue,
		bookmarksdepth=4
}
\makeatother
% -------------------------------------------------------------------------------------------------

% Espaçamentos entre linhas e parágrafos 
% O tamanho do parágrafo é dado por:
\setlength{\parindent}{1.3cm}

% Controle do espaçamento entre um parágrafo e outro:
\setlength{\parskip}{0.2cm}  % tente também \onelineskip
% -------------------------------------------------------------------------------------------------


% compila o indice
\makeindex
% -------------------------------------------------------------------------------------------------


% Início do documento
\begin{document}

% Retira espaço extra obsoleto entre as frases.
\frenchspacing 

% ----------------------------------------------------------
% ELEMENTOS PRÉ-TEXTUAIS
% ----------------------------------------------------------
% \pretextual

% ---
% Capa
% ---
\imprimircapa
% ---

% ---
% Folha de rosto
% (o * indica que haverá a ficha bibliográfica)
% ---
\imprimirfolhaderosto*
% ---

% ---
% Anverso da folha de rosto:
% ---
% -----------------------------------------------------------------------------------------------
%{ ver ser exite necessidade de esta aqui.......??????/
%\ABNTEXchapterfont
%\vspace*{\fill}

%Conforme a ABNT NBR 10719:2011, seção 4.2.1.1.1, o anverso da folha de rosto
%deve conter:

%\begin{alineas}
%  \item nome do órgão ou entidade responsável que solicitou ou gerou o
%   relatório; 
%  \item título do projeto, programa ou plano que o relatório está relacionado;
%  \item título do relatório;
%  \item subtítulo, se houver, deve ser precedido de dois pontos, evidenciando a
%   sua subordinação ao título. O relatório em vários volumes deve ter um título
%   geral. Além deste, cada volume pode ter um título específico; 
%  \item número do volume, se houver mais de um, deve constar em cada folha de
%   rosto a especificação do respectivo volume, em algarismo arábico; 
%  \item código de identificação, se houver, recomenda-se que seja formado
%   pela sigla da instituição, indicação da categoria do relatório, data,
%  indicação do assunto e número sequencial do relatório na série; 
%  \item classificação de segurança. Todos os órgãos, privados ou públicos, que
%   desenvolvam pesquisa de interesse nacional de conteúdo sigiloso, devem
%    informar a classificação adequada, conforme a legislação em vigor; 
%  \item nome do autor ou autor-entidade. O título e a qualificação ou a função
%   do autor podem ser incluídos, pois servem para indicar sua autoridade no
%   assunto. Caso a instituição que solicitou o relatório seja a mesma que o
%   gerou, suprime-se o nome da instituição no campo de autoria; 
%  \item local (cidade) da instituição responsável e/ou solicitante; NOTA: No
%   caso de cidades homônimas, recomenda-se o acréscimo da sigla da unidade da
%   federação.
%  \item ano de publicação, de acordo com o calendário universal (gregoriano),
%  deve ser apresentado em algarismos arábicos.
%\end{alineas}
%\vspace*{\fill}
%}
% -------------------------------------------------------------------------------------------------

% Agradecimentos
% ---
\begin{agradecimentos}
% Não quer mais agradecer? Use o tempo verbal adequado: Agradeço em primeiro... no lugar de Gostaria de agradecer...
Gostaria de agradecer em primeiro lugar a Deus o meu grande guia, por ter me abençoado com a chance de cursar em uma grande faculdade como a CATOLICA DO TOCANTINS - FACTO e ter recebido subsídios tão ricos durante essa jornada, ao professor Marco Antonio Firmino de Sousa pela orientação e motivação para realização deste trabalho, agradeço em especial a minha família por estar ao meu lado na busca por este sonho e também aos meus amigos que souberam conviver e respeitar ainda que nem sempre compartilhássemos as mesmas idéias. E por tudo, a saudade há de ficar.

\end{agradecimentos}
% -------------------------------------------------------------------------------------------------

% ---
% RESUMO
% ---

% resumo na língua vernácula (obrigatório)
\begin{resumo}
% Pela leitura do resumo não consigo compreender o que é ou foi feito. Foco no contexto, na aplicação. As técnicas e ferramentas são coisas de segundo plano. Não abuse de vocabulário coloquial. Seja um pouco mais formal.
Desenvolvimento híbridos é uma área que está em ascensão no âmbito mundial, com isso muitos serviços virão para aproveitar o conceito de mobilidade e acessibilidade. Os dispositivos móveis são os responsáveis por este crescimento, pois eles estão cada vez mais presentes na vida das pessoas. Neste trabalho foi pesquisado formas de desenvolvimento de um aplicativo móvel para o gerenciamento de vendas e compras de uma empresa pet shop. Foi desenvolvido também um estudo técnico sobre alguns sistemas relacionado para que possa abstrair algumas funcionalidades dos mesmos e também melhorar algumas já existentes.

 \vspace{\onelineskip}
    
 \noindent
 \textbf{Palavras-chaves}: Desenvolvimento híbrido. Dispositivos  móveis. Pet shop.
\end{resumo}
% -------------------------------------------------------------------------------------------------

% Inserir somente os ítens que houver conteúdo. Veja que a lista de figuras, tabelas e lista de abreviaturas estão em branco.

% inserir lista de ilustrações
\pdfbookmark[0]{\listfigurename}{lof}
\listoffigures*
\cleardoublepage
% -------------------------------------------------------------------------------------------------


% inserir lista de tabelas
\pdfbookmark[0]{\listtablename}{lot}
\listoftables*
\cleardoublepage
% -------------------------------------------------------------------------------------------------


% inserir lista de abreviaturas e siglas
\begin{siglas}
  \item[Fig.] Area of the $i^{th}$ component
  \item[456] Isto é um número
  \item[123] Isto é outro número
  \item[lauro cesar] este é o meu nome
\end{siglas}
% -------------------------------------------------------------------------------------------------


% inserir lista de símbolos
\begin{simbolos}
  \item[$ \Gamma $] Letra grega Gama
  \item[$ \Lambda $] Lambda
  \item[$ \zeta $] Letra grega minúscula zeta
  \item[$ \in $] Pertence
\end{simbolos}
% -------------------------------------------------------------------------------------------------


% inserir o sumario
%\pdfbookmark[0]{\contentsname}{toc}
%\tableofcontents*
\tableofcontents 
%\cleardoublepage
% -------------------------------------------------------------------------------------------------


% ----------------------------------------------------------
% ELEMENTOS TEXTUAIS
% ----------------------------------------------------------
\textual

% ----------------------------------------------------------
% Capitulo Introdução
% ----------------------------------------------------------
\chapter*[Introdução]{Introdução}
\section*{Contexto}
\addcontentsline{toc}{chapter}{Introdução}
% seja mais profunco, pq a sociedade precisa de mobilidade? pq há tanto dispositivo móvel? Responda isso por aqui.

Nos últimos anos, com o advento dos dispositivos portáteis a nível mundial, houve um crescimento gigantesco de aplicações web mobile, principalmente no meio coorporativo, na utilização de soluções móveis. Devido a tamanha evolução, novas plataformas de desenvolvimento e sistemas operacionais foram criados, aumentando a complexidade e muitas vezes a curva de aprendizado. Um dos maiores problemas com a disparidade entre plataformas é, a manutenção da aplicação, controle de atualizações e custeamento de equipes de desenvolvimento com conhecimentos específicos, como: Android, iOS, BlackBerry, Windows Phone, etc.
As WebApps, são aplicações projetadas para serem executadas em browsers de dispositivos móveis, que acoplada em novas técnicas de desenvolvimento com determinados frameworks estão se tornando os nomeados Web Apps Híbridos. Ou seja, a sua interface gráfica é adaptada para dispositivos com telas menores, utilizando conceitos como o responsive (são web sites que através de técnicas de construção do HTML5 e CSS, é possível termos a versão de um site que se modifique conforme o dispositivo utilizadocom excelente visualização em plataformas e resoluções diferentes). Estas aplicações podem ser acessado de qualquer computador ou dispositivo móvel por meio de um browser conectado à internet, na maior parte, são hospedadas em servidores web, utilizando tecnologias específicas para serem carregadas, em máquinas de “baixo” processamento e normalmente com baixa velocidade de banda da rede.

% Não tenho foco em detalhes técnicos. Nosso curso é atividade meio e é interessante que seu trabalho seja assim. Seu objetivo não é desenvolver técnica, mas uma aplicação para uso em um contexto na sociedade. Explique isso! Detalhe técnico você tratará mais adiante.

% ----------------------------------------------------------
% Capitulo Desenvolvimento
% ----------------------------------------------------------
\chapter*{Desenvolvimento}
% No desenvolvimento você detalha tudo que fez.
% Sessao Objetivo Geral
\section*{Objetivo Geral}
% Objetivo faz parte da introdução.

Este trabalho tem como objetivo realizar um estudo sistemático de conceitos relacionados ao desenvolvimento de aplicações híbridas para dispositivos móveis. Baseado nas necessidades reais de uma empresa de pet shop, que enfrenta problemas como, conciliar todas as atividades da empresa, tendo o cliente que assumi o papel de gerente e veterinário ao mesmo tempo.
Baseado em tal relato o objetivo principal desse trabalho e pesquisar técnicas de como desenvolver um sistema híbrido para diversas plataformas onde possa gerenciar toda a demanda da empresa de pet shop desde um checklist de produtos feito por um vendedor externo até a finalização de um pedido seja na empresa do cliente ou em qualquer outro lugar através dos aplicativos móveis, assim otimizando o tempo de compra dos consumidores.

% -------------------------------------------------------------------------------------------------
% Sessao Objetivo Especifico
\section*{Objetivo Específico}
\begin{itemize}
\item Realizar um estudo sobre o processo de desenvolvimento de aplicações móveis, utilizando tecnologia híbrida.
\item Realizar um estudo teórico dos tipos de aplicativos móveis, suas principais características, vantagens e desvantagens.
\item Realizar um estudo para a construção com praticidade e desempenho de um sistema híbrido que ajudará no controle de uma loja pet shop.
\item Relatar como funciona uma organização deste ramo e as vantagens de um sistema de integração.
\end{itemize}
% -------------------------------------------------------------------------------------------------

%Sessao Motivação
\section*{Motivação}
Em geral obter o maior grau de conhecimento possível sobre os conceitos de desenvolvimento híbrido e futuramente está pondo em pratica todo o conhecimento adquirido neste trabalho. 
% -------------------------------------------------------------------------------------------------

%Sessao justificativa
\section*{Justificativa}
Existem, atualmente, paradigmas principais que norteiam o desenvolvimento de soluções móveis: web, nativo e híbrido. Escolher o melhor é sempre um exercício que exige atenção aos mínimos detalhes.
Baseados nesses paradigmas de desenvolvimento para aplicativos híbridos adotamos esses paradigmas, para uma empresa de pet shop, desenvolvendo uma nova interação entre diversas plataformas, onde vendedores externos e clientes terão acesso a mais informações com mais facilidade, abordando os três paradigmas de desenvolvimento, devido à utilização da linguagem JavaScript, presente em quase todos os frameworks de desenvolvimento, os resultados sempre são significativos em termos de tempo de desenvolvimento do aplicativo, padronização de código, além da praticidade do desenvolvimento.
% -------------------------------------------------------------------------------------------------

% Vamos nos reunir para ordenas todas as seções de seu trabalho.
%Sessao Procedimentos Experimentais
\section*{Procedimentos Experimentais}

Como base para análise utilizarei três software de gerenciamento de vendas para empresas pet shop.

\begin{itemize}
\item Tesche \& Vasconcelos empresa localizada em Poços de Caldas – MG, responsável pelo software PetSystem que trabalha com controle de vendas de produtos para clínica veterinária e pet shop.
\item NetService Consultoria em Sistemas empresa localizada em Gurupi – TO, responsável pelo software PDVtab que trabalha no controle de vendas de produtos para pet shop.
\item Pet Shop Control é um software desenvolvido pela Devsol Softwares localizada em Porto Alegre - RS, que permite administrar sua pet shop e clínica veterinária auxiliando no gerenciamento de todos os setores de sua empresa.
\end{itemize}

\subsection*{Analise Software PetSystem}
desenvolver análise sobre o software .....

\subsection*{Analise Software PDVtab}
desenvolver análise sobre o software .....

%falta o ultimo software para análise.
\subsection*{Analise Software Pet Shop Control}
O pet shop control é um software que permite administrar sua pet shop e clínica veterinária auxiliando no gerenciamento de todos os setores de sua empresa. Através de relatórios de análises financeiras, suas decisões serão tomadas com maior segurança e precisão. Desta maneira, você conquistará uma fatia muito maior do mercado, expandindo seu negócio e aumentando seus lucros. O software pode se adaptar ao seu negócio, preservando o diferencial de mercado e seus métodos de trabalho.
A seguir uma lista com as principais funcionalidades e características relacionada para o entendimento mais profundo.

\textbf{1.	{Recuperação de Base de Dados}}\\
No caso da Pet Shop já possuir um cadastro no banco de dados, você pode restaurar os dados para preservar o histórico das informações e evitar duplicidade de cadastros.
A vantagem nessa funcionalidade e que se a empresa já tem seu banco de dados de outro sistema e decidiu mudar para esse sistema, existe a possibilidade de recuperação da base de dados assim garantindo todos os dados obtido pela empresa ao longo do tempo, minimizando problema como ter que refazer todo o cadastro de cliente por exemplo.

\textbf{2.	{Acesso ao Sistema}}\\
Após a configuração inicial do Pet Shop Control, realizada no primeiro acesso ao sistema, será necessário informar uma identificação de usuário e senha para o acesso completo ao sistema. Uma limitação encontrada e que só consegue fazer novos cadastros de usuário quem tem a senha de administrador total do sistema. A vantagem nesse tipo de restrição e exatamente privar pessoas não autorizadas a manusear o software podendo executar funções que vá a danificar ou até mesmo apagar dados de clientes ou produtos.

\textbf{3.	Pesquisar Serviços}\\
Nesta janela, no campo da busca rápida, é possível pesquisar utilizando como critério o nome do serviço, descrição, nome do grupo, código do cadastro do serviço ou utilizando o nome da categoria do serviço. A vantagem nessa funcionalidade, e listar os agendamentos de serviços efetuados.

\textbf{4.	Configurar um Serviço para Remarcar Automaticamente}\\
Essa funcionalidade habilita que um serviço, como exemplo banho em um determinado animal, possa ser remarcado automaticamente sem nenhuma interversão humana automatizando o trabalho. A vantagem nessa função e que o funcionário não precisara lembrar de um determinado serviço agendado, sendo que ele já está remarcado automaticamente, assim quando um serviço será repetido várias vezes poderá justamente se encaixar nessa função.

\textbf{5.	Pesquisa Inteligente}\\
As janelas de busca inteligente do sistema utilizam um algoritmo de última geração para filtrar os resultados de uma pesquisa. Na janela de busca dos clientes, por exemplo, é possível digitar no campo de busca rápida qualquer um dos seguintes itens para filtrar a busca: nome do proprietário, telefone, celular, e-mail, CPF, RG, código, nome do animal ou endereço. Não é necessário se preocupar com acentuação ou letras maiúsculas e minúsculas nas palavras digitadas, pois o sistema é flexível em relação a estas variações de escrita. A grande vantagem nessa função e da dinamismo e rapidez a pesquisa de um determinado cliente.

\textbf{6.	Lembretes - Quadro de Avisos}\\
O sistema dispõe de um quadro de avisos que apresenta as principais informações utilizadas no dia-a-dia da gestão de sua Pet Shop. A janela de lembretes permite visualizar um resumo das principais informações cadastradas. A vantagem desse tipo de informação e que o funcionário está à par de todos os serviços pendentes por exemplo.

\textbf{7.	Anexar Arquivos ao Cadastro de Animal}\\
Utilize esta ação para anexar exames, arquivos, fotos, vídeos e qualquer tipo de arquivo a um cadastro de animal. A vantagem nessa funcionalidade e que podemos anexar fotos dos animais, isso ajudara em uma futura venda onde os clientes poderão visualizar fotos dos animais e também exames e vídeos.

\textbf{8.	Controle de Vale Compras e Crediário dos Clientes}\\
O sistema permite que você identifique (assinale) o nível de credibilidade de seus clientes. Para isto, você pode utilizar as bandeiras existentes no cadastro do cliente. A grande vantagem e que com essa função e que saberemos exatamente aqueles clientes que mais estão comprando na loja, assim podendo ofertara-los com promoções e brindes e aqueles que possuem algum debito com a empresa, tudo atreves de bandeiras que representam os três tipos de clientes: bom relacionamento de credito, relacionamento mediano de crédito e relacionamento de crédito ruim.

\textbf{9.	Fidelidade do Cliente}\\
Você pode utilizar o sistema para criar um programa de fidelidade do cliente. Desta forma, cada compra do cliente, na loja, poderá acrescentar um valor de bônus para aquisição na loja. Uma limitação, só e capaz de efetuar essa função quem estiver com a senha do administrador. A vantagem nessa operação e que e estipulado um valor para o cliente comprar, se ele atingir esse valor, ganhar descontos em bônus automaticamente no final da compra.

\textbf{10.	Consultar Histórico de Compras de um Fornecedor}\\
Histórico de produtos comprados do fornecedor, com informações como: menor valor pago, maior valor pago e média do valor pago. A vantagem e que esse historio você saberá exatamente diferenças em valores de compras e produtos específicos anteriores.

\textbf{11.	Enviar Mala Direta}\\
Através do envio de mala-direta aos clientes é possível divulgar sua empresa com e-mails marketing, aproximando seus clientes. Uma limitação a essa função e que tanto a empresa como o cliente deve ter um e-mail válido cadastrado no sistema antes de enviar alguma mata direta ao cliente. A principal vantagem dessa funcionalidade e a divulgação da empresa e de seus produtos aos clientes isso através do próprio sistema. Outra vantagem e que o sistema possibilita cadastrar modelos de e-mails, assim quando precisar notificar um cliente que sua conta está vencendo por exemplo, já tem um e-mail de vencimento como modelo pronto pra ser enviado.
 

Com algumas das principais funcionalidades descritas acima, observamos com detalhes o funcionamento do software Pet Shop Control. Funcionalidades que auxiliam funcionários a trabalhar de forma correta e dando mais agilidade no processo de gerenciamento da organização.





%CONTINUA........


% Capitulo com exemplos de comandos inseridos de arquivo externo 
% ----------------------------------------------------------

\include{abntex2-modelo-include-comandos}


% ----------------------------------------------------------
% Parte de revisãod e literatura
% ----------------------------------------------------------
\part{Resultados}

% ---
% Capitulo de revisão de literatura
% ---
\chapter{Lorem ipsum dolor sit amet}

% ---
\section{Aliquam vestibulum fringilla lorem}
% ---

\lipsum[1]

\lipsum[2-3]

% ---
% Finaliza a parte no bookmark do PDF, para que se inicie o bookmark na raiz
% ---
\bookmarksetup{startatroot}% 
% ---

% ---
% Conclusão
% ---
\chapter*[Conclusão]{Conclusão}
\addcontentsline{toc}{chapter}{Conclusão}

\lipsum[31-33]

% ----------------------------------------------------------
% ELEMENTOS PÓS-TEXTUAIS
% ----------------------------------------------------------
\postextual

% ----------------------------------------------------------
% Referências bibliográficas
% ----------------------------------------------------------
\bibliography{abntex2-modelo-references}

% ----------------------------------------------------------
% Glossário
% ----------------------------------------------------------
%
% Consulte o manual da classe abntex2 para orientações sobre o glossário.
%
%\glossary

% ----------------------------------------------------------
% Apêndices
% ----------------------------------------------------------

% ---
% Inicia os apêndices
% ---
\begin{apendicesenv}

% Imprime uma página indicando o início dos apêndices
\partapendices

% ----------------------------------------------------------
\chapter{Quisque libero justo}
% ----------------------------------------------------------

\lipsum[50]

% ----------------------------------------------------------
\chapter{Nullam elementum urna vel imperdiet sodales elit ipsum pharetra ligula
ac pretium ante justo a nulla curabitur tristique arcu eu metus}
% ----------------------------------------------------------
\lipsum[55-57]

\end{apendicesenv}
% ---


% ----------------------------------------------------------
% Anexos
% ----------------------------------------------------------

% ---
% Inicia os anexos
% ---
\begin{anexosenv}

% Imprime uma página indicando o início dos anexos
\partanexos

% ---
\chapter{Morbi ultrices rutrum lorem.}
% ---
\lipsum[30]

% ---
\chapter{Cras non urna sed feugiat cum sociis natoque penatibus et magnis dis
parturient montes nascetur ridiculus mus}
% ---

\lipsum[31]

% ---
\chapter{Fusce facilisis lacinia dui}
% ---

\lipsum[32]

\end{anexosenv}

%---------------------------------------------------------------------
% INDICE REMISSIVO
%---------------------------------------------------------------------

\printindex

%---------------------------------------------------------------------
% Formulário de Identificação (opcional)
%---------------------------------------------------------------------
\chapter*[Formulário de Identificação]{Formulário de Identificação}
\addcontentsline{toc}{chapter}{Exemplo de Formulário de Identificação}
\label{formulado-identificacao}

Exemplo de Formulário de Identificação, compatível com o Anexo A (informativo)
da ABNT NBR 10719:2011. Este formulário não é um anexo. Conforme definido na
norma, ele é o último elemento pós-textual e opcional do relatório.

\bigskip

\begin{tabular}{|p{9cm}|p{5cm}|}
\hline
\multicolumn{2}{|c|}{\textbf{\large Dados do Relatório Técnico e/ou científico}}\\
\hline
\multirow{4}{10cm}[24pt]{Título e subtítulo}& Classificação de segurança\\
                   & \\
                   \cline{2-2}
                   & No.\\
                   & \\
				
\hline
Tipo de relatório & Data\\
\hline
Título do projeto/programa/plano & No.\\
\hline
\multicolumn{2}{|l|}{Autor(es)} \\
\hline
\multicolumn{2}{|l|}{Instituição executora e endereço completo} \\
\hline
\multicolumn{2}{|l|}{Instituição patrocinadora e endereço completo} \\
\hline
\multicolumn{2}{|l|}{Resumo}\\[3cm]
\hline
\multicolumn{2}{|l|}{Palavras-chave/descritores}\\
\hline
\multicolumn{2}{|l|}{
Edição \hfill No. de páginas \hfill No. do volume \hfill Nº de classificação \phantom{XXXX}} \\
\hline
\multicolumn{2}{|l|}{
ISSN \hfill \hfill Tiragem \hfill Preço \phantom{XXXXXXXX}} \\
\hline
\multicolumn{2}{|l|}{Distribuidor} \\
\hline
\multicolumn{2}{|l|}{Observações/notas}\\[3cm]
\hline
\end{tabular}

\end{document}